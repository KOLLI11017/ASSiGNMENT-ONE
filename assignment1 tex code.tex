\documentclass[11pt]{article}
\usepackage{amsmath}
\usepackage{amssymb}
\usepackage{siunitx}
\begin{document}
	\providecommand{\norm}[1]{\lVert#1\rVert}
	\newcommand{\myvec}[1]{ \left[\begin{array}{c}#1\end{array}\right] }
	\textnormal{\hspace{-8ex}(Q). Find $\theta$ and P if ($\sqrt{3}$\hspace{2mm} 1)x = -2 \hspace{1mm}is equivalent to ($\cos$$\theta$\hspace{2mm} $\sin$$\theta$)x = P ?}\\[3ex]
	\textnormal{Given,\hspace{3mm}($\sqrt{3}$\hspace{2mm} 1)x = -2 \hspace{1mm}is equivalent to ($\cos$$\theta$\hspace{2mm} $\sin$$\theta$)x = P}\\[1.5em]
	\textnormal{The given equation of line is,}
	\begin{flalign}
		n^T\ x = c\\[1.5ex]
		where \  \vec{n} = \myvec{\sqrt{3}&1} and\ c = -2
	\end{flalign}
	\textnormal{}
	\textnormal{Now obtain a new equation,}
	\begin{flalign}
		\frac{\vec{n}}{\norm{\vec{n}}} = \frac{c}{\norm{\vec{n}}} \hspace{6mm} where\ \norm{\vec{n}}\ is\ the\ norm\ of\ the\ \vec{n}\\[1.5ex]
		\implies \vec{u}x = P\ \hspace{4mm}where\ u = \myvec{\cos \theta  &   sin \theta}\ and\ \vec{P} = \frac{c}{\norm{\vec{n}}}
	\end{flalign}
	\textnormal{Substituting the values of $\vec{n}$ and c, we get}
	\begin{flalign}
		\norm{\vec{n}} = 2\\[1.5ex]
		\myvec{\sqrt{3}/2 & 1/2}x = -1\\[1.5ex]
	\end{flalign}	
	\textnormal{From (4) and (5). we derive,}
	\begin{flalign}
		\cos\theta = \sqrt{3}/2\ ;\hspace{3mm} \sin\theta = 1/2\\[1.5ex] P = -1\\[1.5ex]
		\therefore\ \hspace{1mm} \theta = cos^{-1}(\frac{\sqrt{3}}{2}) = sin^{-1}(\frac{1}{2}) = \ang{30}\hspace{3mm} and\hspace{3mm} P = -1
	\end{flalign}
\end{document}
